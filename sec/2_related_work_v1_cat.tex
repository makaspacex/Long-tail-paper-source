% !TeX root = ../main.tex
\section{Related Works}
\label{sec:formatting}
In this section, we review previous works on layout analysis and text recognition. For document text recognition, we focus on character-based methods.


\subsection{Historical Text Recognition}
% Too much background, consider dropping this paragraph
Document digitization systems protect printed paper documents from direct manipulation and facilitate consultation, exchange and remote access. Specifically, text recognition is one of its two main stages together with layout analysis~\cite{jla}.

Historical text recognition methods can be divided into character-based methods and sequence-based methods. 
Character-based recognition methods typically involve locating individual characters, recognizing them, and grouping them into lines of text~\cite{papytwin}.

Among the three steps, the single character recognition step is mostly researched, due to it faces challenging problems including broken character~\cite{broken}, wild writing styles~\cite{obc306}, large class numbers with long-tailed distribution~\cite{fewran}, or on the extreme end novel characters that are not covered by the training samples~\cite{hde,ligarature}.

Due to character-level annotations is usually more expensive to obtain,  sequence-based methods, which trains on line-level images and annotations are proposed~\cite{eccvfork,jinic21}. Still, they faces similar challenges posed to character level counterparts. In this work, we focus on the long-tailed challenge in the historical text recognition tasks.

\subsection{The Long Tailed Distribution Problem}


In real life, data, specifically training data, often have imbalanced instance count for different labels, whose distributions exhibits broader characteristics than the standard positive land distribution, called long-tail distributions~\cite{}. 

Specifically, a small number of individuals make significant contributions, resulting in the minority class dominating the data set (called the head class), while the majority class contains only a few data samples (called the tail class). 


Zero-shot learning~\cite{gzsl-survey}, as an extreme case of long-tailed problem


\subsection{Long Tails in Historical Text Recognition}

Long-tail distribution is very common in ancient text recognition (See Fig~\ref{fig:moretailed}). 
The important reasons are: 1) The long-tail characteristics of human language itself. 2) The number of ancient books is not large. The long-tail problem is one of the important challenges often encountered in historical document text recognition tasks.